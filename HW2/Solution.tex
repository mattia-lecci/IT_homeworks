\documentclass[12pt]{article}

\usepackage[hmargin=2.5cm, vmargin={3cm,3cm}, a4paper]{geometry}
\usepackage{amsmath,amssymb,amsthm,mathrsfs}
\usepackage{epsfig,epsf,subfigure,graphicx,graphics}
\usepackage{url,enumerate}
\usepackage{physics,dsfont,ragged2e}
\usepackage{algorithm2e}
\usepackage{glossaries}
\graphicspath{{fig/}}
\usepackage{fancyhdr}
\setlength{\headheight}{12pt}
\pagestyle{fancyplain}

\numberwithin{equation}{section}
\newtheorem{theorem}{Theorem}

% definitions/operators
\DeclareMathOperator*{\argmax}{arg\,max\;}
\DeclareMathOperator*{\argmin}{arg\,min\;}
\newcommand{\conv}[1]{\operatorname{Conv} \qty(#1)}
\newcommand{\x}[1]{d(X^n(#1),Y^n)}
\newcommand{\E}{\mathcal{E}}
\newcommand{\M}{\mathcal{M}}
\newcommand{\K}{\mathcal{K}}
\newcommand{\LL}{\mathcal{L}}
\newcommand{\X}{\mathcal{X}}
\newcommand{\Y}{\mathcal{Y}}
\newcommand{\Z}{\mathcal{Z}}
\newcommand{\typicalitySet}{\mathcal{T}_\varepsilon^{(n)}}
\newcommand{\Pen}{\ensuremath{ P_e^{(n)} } }
\newcommand{\indicator}{\mathds{1}}
\newcommand{\eqtext}[1]{\ensuremath{\stackrel{#1}{=}}} % equal sign with some text over it
\newcommand{\mysection}[1]{\renewcommand{\thesection}{#1}\section}
\renewcommand{\thesubsection}{\thesection.\alph{subsection}} % to use letters for subsections
\newcommand\independent{\protect\mathpalette{\protect\independenT}{\perp}} % see next one
\def\independenT#1#2{\mathrel{\rlap{$#1#2$}\mkern2mu{#1#2}}} % to have independet symbol


\rhead{}
\lhead{}
\chead{{\it Network Information Theory}}
\lfoot{}
\cfoot{\thepage}
\rfoot{}
\title{Solution of Homework \#2}
\author{Mattia Lecci \and Federico Mason \and Paolo Testolina}
\begin{document}
\maketitle
\thispagestyle{fancyplain}
\flushleft


%% execises
\justify
For the solution of these homeworks we will always consider the function $\log(x)$ as $\log_2(x)$ for simplicity.

\mysection{10.1}{Problem 10.1}
\section{Problem 3.13}

Is given a Gaussian product channel $Y_j = g_j \cdot X_j + Z_j$ $j\in \{1,2\}$ with $g_1 < g_2$ and average power constraint $P$. We want to know above what power $P^*$ we start to use both the channels and what are the features of the \textit{energy-per-bit-rate function} $E_b(R)$.

\subsection{Optimal power allocation}


We notiche that to achieve the maximum capacity from the channel we have to solve the following optimization problem.

\begin{equation}
\begin{gathered}
	\max_{P_j} \sum_{j=1}^2 \frac{1}{2} \cdot \log(1+g_j^2 \cdot P_j) \quad subject \quad to\\
	-P_j \neq 0 \quad j\in \{1,2\} \\
	\sum_{j=1}^2 P_j - P = 0
\end{gathered}
\label{optproblem}
\end{equation}

In order to solve \eqref{optproblem} we build the following equation.

\begin{equation}
	\nabla_{P_j}\Big\{-\sum_{j=1}^2 \frac{1}{2} \cdot \log(1+g_j^2 \cdot P_j) + \sum_{j=1}^2 \lambda_j \cdot (-P_j) + \nu \cdot \Big(\sum_{j=1}^2 P_j - P\Big)  \Big\} = 0
	\label{f0}
\end{equation}

Solving \eqref{f0} we obtain what follows.

\begin{equation}
	\begin{gathered}
		\frac{1}{2} \cdot \frac{1}{P_j+\frac{1}{g_j^2}}+\lambda_j-\nu=0 \\
		\Rightarrow \frac{1}{2} \cdot \frac{1}{P_j+\frac{1}{g_j^2}} \leq \nu = \frac{1}{2\mu}
	\end{gathered}
\end{equation}

\begin{equation}
	\begin{gathered}
		\sum_{j=1}^2 \max\Big\{\mu - \frac{1}{g_j^2},0\Big\} = P \\
		\Rightarrow \mu = \frac{P}{2}+\frac{1}{2g_1^2} + \frac{1}{2g_2^2}
	\end{gathered}
\end{equation}

\begin{equation}
	\begin{gathered}
	 P_j = \max\Big\{\mu - \frac{1}{g_j^2},0\Big\} \\
	 \Rightarrow P_1= \max\Big\{\frac{P}{2} + \frac{1}{2g_2^2} - \frac{1}{2g_1^2},0\Big\} = \begin{cases}
	  \frac{P}{2} + \frac{1}{2g_2^2} - \frac{1}{2g_1^2} \quad P \geq \frac{1}{g_1^2} - \frac{1}{g_2^2} \\
		0 \quad P < \frac{1}{g_1^2} - \frac{1}{g_2^2}
	 \end{cases} \\
	 \Rightarrow P_2= \max\Big\{\frac{P}{2} + \frac{1}{2g_1^2} - \frac{1}{2g_2^2},0\Big\} = \frac{P}{2} + \frac{1}{2g_1^2} - \frac{1}{2g_2^2} \quad \forall \quad P \geq 0
	\end{gathered}
\end{equation}

We conclude that the second channel is opened for every amount of allocated power $P$ while the first channel is opened only if and only if \eqref{Pcondition} is verified.

\begin{equation}
	P \geq \frac{1}{g_1^2} - \frac{1}{g_2^2}
	\label{Pcondition}
\end{equation}

\subsection{Energy-per-bit-rate function computation}

Now we want to compute the \textit{energy-per-bit-rate function} $E_b(R)$ in the scenario previously described. We know that $P = R \cdot E$ where $R$ is the \textit{bit-rate} of the channel and $E$ is the used energy for every bit. In addiction we know that depending of what amount of power is avaliable, the \textit{bit-rate} is restricted as shown in \eqref{1case} and \eqref{2case}.

\begin{equation}
		R \leq \frac{1}{2} \cdot \log\big( 1 + g_2^2P \big) \Leftrightarrow P < \frac{1}{g_1^2} - \frac{1}{g_2^2}
		\label{1case}
\end{equation}


\begin{equation}
		R \leq \frac{1}{2} \cdot \sum_{j=1}^2 \log\Big(\frac{g_j^2}{2} \cdot \Big(P+\frac{1}{g_1^2} + \frac{1}{g_2^2}\Big)\Big) \Leftrightarrow P \geq \frac{1}{g_1^2} - \frac{1}{g_2^2}
		\label{2case}
\end{equation}

Supponing to achieve the maximum rate in both \eqref{1case} and \eqref{2case} we can compute R in function of P and then write again the boundaries condition as follows.

\begin{equation}
	\begin{gathered}
		R \leq \frac{1}{2} \cdot \log\big( 1 + g_2^2P \big) \\
		\Rightarrow \frac{2^{2R}-1}{g_2^2} \leq P < \frac{1}{g_1^2} - \frac{1}{g_2^2} \\
		\Rightarrow R < \frac{1}{2} \log\Big(\frac{g_2^2}{g_1^2}\Big)
	\end{gathered}
	\label{bound1}
\end{equation}

\begin{equation}
	\begin{gathered}
		R \leq \frac{1}{2} \cdot \sum_{j=1}^2 \log\Big(\frac{g_j^2}{2} \cdot \Big(P+\frac{1}{g_1^2} + \frac{1}{g_2^2}\Big)\Big) \\
		\Rightarrow P \geq \frac{2^{R+1}}{g_1g_2} -\frac{1}{g_1^2} - \frac{1}{g_2^2}
	\end{gathered}
	\label{bound2}
\end{equation}


Let's first consider the case where only one channel is opened. Sobstituting $P=E \cdot R$ in \eqref{1case} we get what follows.

\begin{equation}
	\begin{gathered}
		R \leq \frac{1}{2} \cdot \log\big( 1 + g_2^2 E R \big) \\
		\Rightarrow E \geq \frac{1}{R}\Big(\frac{2^{2R}-1}{g_2^2}\Big)
	\end{gathered}
\end{equation}

Let's now consider the case where both the channels are opened. Sobstituiting $P=E \cdot R$ in \eqref{2case} we get what follows.

\begin{equation}
	\begin{gathered}
		R \leq \frac{1}{2} \cdot \sum_{j=1}^2 \log\Big(\frac{g_j^2}{2} \cdot \Big(ER+\frac{1}{g_1^2} + \frac{1}{g_2^2}\Big)\Big) \\
		\Rightarrow R \leq \frac{1}{2} \cdot \log\Big(\frac{g_1^2 g_2^2}{4} \cdot \Big(ER+\frac{1}{g_1^2} + \frac{1}{g_2^2}\Big)^2\Big) \\
		\Rightarrow E \geq \frac{1}{R} \Big(\frac{2^{R+1}}{g_1g_2} -\frac{1}{g_1^2} - \frac{1}{g_2^2} \Big)
	\end{gathered}
\end{equation}

We conclude that \textit{energy-per-bit rate function} is equal to the minimum energy-per-bit required for every value of R as we can see from \eqref{epbfunction}.

\begin{equation} E_b(R)=
	\begin{cases}
		\frac{1}{R}\Big(\frac{2^{2R}-1}{g_2^2}\Big) \quad P < \frac{1}{g_1^2} - \frac{1}{g_2^2}  \\
		\frac{1}{R} \Big(\frac{2^{R+1}}{g_1g_2} -\frac{1}{g_1^2} - \frac{1}{g_2^2} \Big)  \quad P \geq \frac{1}{g_1^2} - \frac{1}{g_2^2}
	\end{cases}
	\label{epbfunction}
\end{equation}

Now we want to compute the minimum energy-per-bit as $R \rightarrow 0$.


\begin{equation}
	\lim_{R \rightarrow 0} E_b(R) =
		\lim_{R \rightarrow 0} \Big(\frac{2^{2R}-1}{g_2^2 R}\Big) = \lim_{R \rightarrow 0} \frac{ \frac{d(2^{2R}-1)}{dR}} {\frac{dg_2^2 R}{dR}}=\frac{2\ln{2}}{g_2^2} \quad P < \frac{1}{g_1^2} - \frac{1}{g_2^2}
\end{equation}

\begin{equation}
		\lim_{R \rightarrow 0} E_b(R) =
		\lim_{R \rightarrow 0} \frac{1}{R} \Big(\frac{2^{R+1}}{g_1g_2} -\frac{1}{g_1^2} - \frac{1}{g_2^2} \Big) = \lim_{R \rightarrow 0} \frac{ \frac{d(2^{R+1})}{dR}} {\frac{dg_1g_2 R}{dR}}=\frac{2\ln{2}}{g_1g_2}  \quad P \geq \frac{1}{g_1^2} - \frac{1}{g_2^2}
\end{equation}

We conclude that the minimum energy-per-bit as $R \rightarrow 0$ is equal to \eqref{emin}.

\begin{equation}
	\lim_{R \rightarrow 0} E_b(R) = \begin{cases}
		\frac{2\ln{2}}{g_2^2} \quad P < \frac{1}{g_1^2} - \frac{1}{g_2^2} \\
		\frac{2\ln{2}}{g_1g_2} \quad P \geq \frac{1}{g_1^2} - \frac{1}{g_2^2}
\end{cases}
\label{emin}
\end{equation}

\section{Problem 3.5}

\subsection{3.5(a)}
Let's define
%
\begin{equation}
N(m) \triangleq \left|\{ i: y_i=x_i(m) \} \right|
\end{equation}
%
Note that it is strictly related to to the hamming distance, in fact $d(x^n(m),y^n) = n-N(m)$.

In the case of a memoryless channel, the MLD works as follows:
%
\begin{equation}
\hat{m} = \argmax_m \prod_{i=1}^{n} p_{Y|X}(y_i|x_i(m)) = \argmax_m (1-p)^{N(m)} p^{n-N(m)}
\end{equation}
%
where the last equality comes from the fact that the Binary Symmetric Channel (BSC) only cares about whether a symbol is correctly or incorrectly transmitted. Assuming that $x^n(m)$ is sent and $y^n$ is received, there will be $N(m)$ correctly transmitted bits (each w.p. $(1-p)$ and $n-N(m)$ errors (each w.p. $p$).

Using the natural logarithm (which is a strictly monotonically increasing function) we obtain
%
\begin{align}
\begin{split}
\hat{m} =& \argmax_m N(m)\log(1-p) + (n-N(m)) \log p\\
\eqtext{(a)}& \argmax_m N(m)\log(1-p) -N(m) \log p\\
=& \argmax_m N(m)\log\left( \frac{1-p}{p} \right)\\
\eqtext{(b)}& \argmax_m N(m)\\
\eqtext{(c)}& \argmin_m n-N(m)\\
=& \argmin_m d( x^n(m),y^n)
\end{split}
\end{align}
%
where in $(a)$ we noted that the factor $n \log p$ doesn't depend on $m$, hence it can be neglected, in $(b)$ again we neglect the constant positive term (by assumption $p<\frac{1}{2}$), in $(c)$ we apply a strictly monotonically decreasing function ($x \rightarrow n-x$ with a given $n$), hence we need to search for a minimum instead of a maximum. The last equality has already been discussed at the beginning of the section.

\subsection{3.5(b)}

\subsection{3.5(c)}

First, we want to show that
%
\begin{equation}\label{eq:3.5c_goal}
\Pr[d(X^n(1),Y^n)>n(p+\varepsilon) | M=1] \xrightarrow{n\rightarrow \infty} 0
\end{equation}

Recalling the Weak Law of Large Number for iid random variables
%
\begin{equation}
\lim_{n \rightarrow \infty} \Pr \qty( \qty|\frac{1}{n} \sum_{i=1}^{n} X_i - E[X]|> \varepsilon ) = 0
\end{equation}
%
we see that it's closely related to our case of study. In fact consider
%
\begin{equation}
E[d(X^n(1),Y^n)] = E\qty[ \sum_{i=1}^{n} \mathds{1}[Y_i(1) \neq Y_i] ] = \sum_{i=1}^n E[ \mathds{1}[Y_i(1) \neq Y_i] ] = \sum_{i=1}^n p = np
\end{equation}

From here and from Eq.~\eqref{eq:3.5c_goal} we cam rearrange terms
%
\begin{equation}
\Pr \qty[ \frac{1}{n} d(X^n(1),Y^n) - p > \varepsilon |M=1] \leq \Pr \qty[ \qty| \frac{1}{n} d(X^n(1),Y^n) - p| > \varepsilon] \xrightarrow{n\rightarrow \infty} 0
\end{equation}
%
where the inequality comes from the fact that, considering $X$ a generic random variable and $a>0$ a number, the event $\{|X|>a\}$ can be written as $\{X>a\} \cup \{X<-a\}$ (which are clearly disjoint events). Hence $\Pr[|X|>a] = \Pr[X>a] + \Pr[X<-a] \geq \Pr[X>a]$ (since probabilities are non-negative quantities).
\mysection{5.15}{Problem 5.15}

We consider two different Gaussian channels. The first channel is a Gaussian BC defined by $Y_1 = g_1 X + Z_1$  and $Y_2 = g_2 X + Z_2$ with $Z_i \sim N(0,1)$ $i \in \{1,2\}$ and average power constraint $P$ on $X$ as shown in \eqref{eq:powcon1}.

\begin{equation}
	\sum_{i=1}^n x_i(m)^2 \leq nP \quad m \in [1:2^{nR}]
	\label{eq:powcon1}
\end{equation}

The second channel is a Gaussian MAC defined by $Y = g_1 X_1 + g_2 X_2 + Z$ with $Z \sim N(0,1)$ and average power constraint $P$ on $X_1$ and $X_2$ as shown in \eqref{eq:powcon2}.

\begin{equation}
	\sum_{i=1}^n \left( x_{1i}(m_1)^2 + x_{2i}(m_2)^2 \right) \leq nP \quad (m_1,m_2) \in [1:2^{nR_1}] \times [1:2^{nR_2}]
	\label{eq:powcon2}
\end{equation}

\subsection{Characterize capacity regions}

Our aim is to characterize the two capacity regions $\C_{BC}(R_1, R_2)$ and $\C_{MAC}(R_1, R_2)$ of the two channels. We define the $C(x)$ function as $C(x)=\frac{1}{2}log(1+x)$. We assume $g_1 \geq g_2$ and we call $S_1=g_1^2P$ and $S_2=g_2^2P$. The capacity region $\C_{BC}(R_1, R_2)$ of the Gaussian BC is defined by \eqref{eq:capBC} for some $\alpha \in [0,1]$.

\begin{equation}
	\begin{gathered}
		R_1 \leq C(\alpha S_1) \\
		R_2 \leq C\left(\frac{(1-\alpha)S_2}{1+\alpha S_2}\right)
	\end{gathered}
	\label{eq:capBC}
\end{equation}

The capacity region $\C_{MAC}(R_1, R_2)$ of the Gaussian MAC is defined by \eqref{eq:capMAC} for some $\beta \in [0,1]$.

\begin{equation}
	\begin{gathered}
		R_1 \leq C(\beta S_1) \\
		R_2 \leq C((1-\beta) S_2) \\
		R_1 + R_2 \leq C(\beta S_1 + (1-\beta)S_2)
	\end{gathered}
	\label{eq:capMAC}
\end{equation}

We notice that the variable $\beta \in [0,1]$ establishes the power allocation in Gaussian MAC as we can see in \eqref{eq:powcon3}.

\begin{equation}
	\begin{gathered}
		\sum_{i=1}^n x_{1i}(m_1)^2 \leq n \beta P \quad m_1 \in [1:2^{nR_1}]\\
		\sum_{i=1}^n x_{2i}(m_2)^2 \leq n (1- \beta ) P \quad m_2 \in [1:2^{nR_2}]
	\end{gathered}
	\label{eq:powcon3}
\end{equation}

\subsection{Regions equality}

Our aim is prove that the two capacity regions $\C_{BC}(R_1, R_2)$ and $\C_{MAC}(R_1, R_2)$ determine the same area in the $(R_1, R_2)$ space. In order to do so we notice that both the regions $\C_{BC}(R_1, R_2)$ and $\C_{MAC}(R_1, R_2)$ are convex. Given the geometric structure of the two regions we can state that to prove that one region is a subset of the other one we only need to prove that the corner points of the first region are contained in the second region.

Given $\alpha \in [0,1]$ the only corner point of $\C_{BC}$ is given by \eqref{eq:cornerBC}.

\begin{equation}
	P_{BC} = \left( C(\alpha S_1) , C \left( \frac{(1-\alpha)S_2}{1+\alpha S_2} \right) \right)
	\label{eq:cornerBC}
\end{equation}

Given $\beta \in [0,1]$ the two corner points of $\C_{MAC}$ are given by \eqref{eq:cornerMAC}.

\begin{equation}
	\begin{gathered}
		P_{MAC}^1 = (\left( C (\beta S_1 + (1-\beta)S_2) - C((1-\beta)S_2) , C \left( (1-\beta)S_2 \right) \right)) \\ P_{MAC}^2 = (\left(C \left(\beta S_1 \right), C (\beta S_1 + (1-\beta)S_2) - C(\beta S_1) \right))
	\end{gathered}
	\label{eq:cornerMAC}
\end{equation}

Now we want to show that $\C_{BC} \subseteq \C_{MAC}$. We put $\beta = \alpha(1+S_2)/(1+\alpha S_2)$ and we obtain what follows.

\begin{gather*}
	\beta = \frac{\alpha(1+S_2)}{1+\alpha S_2} \quad  \alpha = \frac{\beta}{1+ (1-\beta) S_2} \Rightarrow \\
	\left( C(\alpha S_1) , C \left( \frac{(1-\alpha)S_2}{1+\alpha S_2} \right) \right) = \left( C (\beta S_1 + (1-\beta)S_2) - C((1-\beta)S_2) , C \left( (1-\beta)S_2 \right) \right)
\end{gather*}

We have found a value of $\beta$ for which $P_{BC} = P_{MAC}^1$. The only corner point of $\C_{BC}$ is contained in $\C_{MAC}$ and therefore $\C_{BC} \subseteq \C_{MAC}$.

Now we want to show that $\C_{MAC} \subseteq \C_{BC}$. First we put $\alpha = \beta / (1+(1-\beta)S_2)$ and we obtain what follows.

\begin{gather*}
	\alpha = \frac{\beta}{1+ (1-\beta) S_2} \quad \beta = \frac{\alpha(1+S_2)}{1+\alpha S_2}  \Rightarrow \\
	\left( C (\beta S_1 + (1-\beta)S_2) - C((1-\beta)S_2) , C \left( (1-\beta)S_2 \right) \right)
	= \left( C(\alpha S_1) , C \left( \frac{(1-\alpha)S_2}{1+\alpha S_2} \right) \right)
\end{gather*}

We have found a value of $\alpha$ for which $P_{MAC}^1 = P_{BC}$. Then we put $\alpha = \beta$ and we obtain what follows.

\begin{gather*}
	\alpha = \beta \Rightarrow \\
	(\left( , C \left(\beta S_1 \right), C (\beta S_1 + (1-\beta)S_2) - C(\beta S_1) \right)) =  \left( C \left(\alpha S_1 \right), C \left( \frac{(1-\alpha)S_2}{1 + \alpha S_1}\right) \right)
\end{gather*}

We remember that $g_1 \geq g_2$. This implies that $S_1 \geq S_2$ and therefore we obtain what follows.

\begin{equation*}
	\frac{(1-\alpha)S_2}{1 + \alpha S_1} \leq \frac{(1-\alpha)S_2}{1 + \alpha S_2}
\end{equation*}

We proved that $P_{MAC}^2$ has one coordinate identical to $P_{BC}$ and the other cordinate equal or lower. We have found a value of $\alpha$ for which $P_{MAC}^1 \in \C_{BC}$. Both the corners point of $\C_{MAC}$ are contained in $\C_{BC}$ and therefore $\C_{MAC} \subseteq \C_{BC}$. Since we proved that $\C_{BC} \subseteq \C_{MAC}$ and that $\C_{MAC} \subseteq \C_{BC}$, the proof of $\C_{BC}=\C_{MAC}$ is given.

\subsection{Boundary MAC achievability}

Our aim is demostrate that every point $(R_1,R_2)$ on the boundary of $\C_{MAC}$ is achievable using random coding and succesive cancellation deconding. First we rember that the two capacity regions coincide $\C_{BC}=\C_{MAC}$. Therefore every boundary point of $\C_{MAC}$ coincide with a boundary point of $\C_{BC}$. The boundary points of $\C_{BC}$ can be always defined as seen in \eqref{eq:cornerBC} with $\alpha \in [0,1]$. We remember that $P_{BC} = P_{MAC}^1$ for $\beta = \alpha(1+S_2)/(1+\alpha S_2)$. We conclude that every boundary point of $\C_{MAC}$ can be written as shown in \eqref{eq:boundmac} for $\beta \in [0,1]$.

\begin{equation}
	(\left( C (\beta S_1 + (1-\beta)S_2) - C((1-\beta)S_2) , C \left( (1-\beta)S_2 \right) \right))
	\label{eq:boundmac}
\end{equation}

Therefore to prove that all the boundary points of $\C_{MAC}$ are achievable, we only need to prove that the corner point $P_{MAC}^1$ is achievable using $\beta P$ and $(1-\beta) P$ as constraints of the two trasmitter. We have already state during the course that the corners point of the capacity region of a MAC channel are achievable using random coding and succesive cancellation deconding and without use time sharing. Therefore the proof is given.

\subsection{Boundary BC achievability}

Our aim is demostrate that the boundary points of $\C_{BC}$ is achievable using the same sequence of codes that achieves the boundary points of $\C_{MAC}$. Honestely we believe that this question is not clear, since Gaussian MAC and Gaussian BC represent two completely different scenarios and they cannot works with same sequence of codes. However we will solve the exercise finding a sequence codes for the Gaussian BAC that is obtained by the sequence of codes for the Gaussian BC; in particular we will a sequence of codes for the Gaussian BC with the same power constraint of the  Gaussian BC.


Suppose to have sequence of code $(2^{nR_1}, 2^{nR_2}, n)$ that achieves the boundary points of $\C_{MAC}$. As we saw in the previous point of the exercise, this means that the sequence of codes is able to achieve the point \eqref{eq:boundmac}. We assume that the power allocation is enhanced so that $x_1^n$ obtains $n \beta P$ while $x_2^n$ obtains $n (1-\beta) P$. We also assume that the chosen codes are uncorrelated for large $n$ as we can see in \eqref{eq:uncor}.

\begin{equation}
	\frac{1}{n}\sum_{i=1}^n x_{1i}(m_1) x_{2i} (m_2)= 0 \quad \forall (m_1,m_2) \in [1:2^{nR_1}]\times[1:2^{nR_2}]
	\label{eq:uncor}
\end{equation}

Now we want to build a code sequence $(2^{n(R_1+R_2)},n)$ for the Gaussian BC starting from the given codes sequence for the Gaussian MAC. We define a new BC codebook in \eqref{eq:codebook}.

\begin{equation}
	x_i(m_1,m_2)= \sqrt{\frac{\alpha}{\beta}}x_{1i}(m_1)+\sqrt{\frac{1-\alpha}{1-\beta}}x_{2i}(m_2) \quad \forall i \in [1:n]
	\label{eq:codebook}
\end{equation}

We notice that the BC codebook requires the same amount of power of the MAC codebook, as we can seen in \eqref{eq:powerallBC}.

\begin{equation}
	\begin{aligned}
		\sum_{i=1}^n x_i^2(m_1,m_2) = & \sum_{i=1}^n \frac{\alpha}{\beta} x_{1i}^2(m_1) + \sum_{i=1}^n \frac{1-\alpha}{1-\beta} x_{2i}^2(m_2) + \sum_{i=1}^n \sqrt{\frac{\alpha(1-\alpha)}{\beta(1-\beta)}} x_{1i}(m_1) x_{2i}(m_2)\\
		= & \alpha nP + (1-\alpha) nP = nP
	\end{aligned}
	\label{eq:powerallBC}
\end{equation}

In \eqref{eq:powerallBC} we have used the fact that $x_1^n$ and $x_2^n$ are uncorrelated for large values of $n$. Obviously the working of the BC system, is base on sending $x^n(m_1,m_2)$  as ecnoded message of $(m_1,m_2)$.

From the theory of Gaussian Broadcast channel, we know that boundary achievability conditions are given by \eqref{eq:bcachiev}.

\begin{equation}
\begin{gathered}
	R_1 \leq C(\alpha S_1) \\
	R_2  \leq C \left( \frac{(1-\alpha )S_1} {1+\alpha S_1} \right)
\end{gathered}
\label{eq:bcachiev}
\end{equation}

We chose to use $\alpha = \frac{\beta}{1+ (1-\beta) S_2}$, obtaining \eqref{eq:macachiev}.

\begin{equation}
\begin{gathered}
	R_1 \leq C (\beta S_1 + (1-\beta)S_2) - C((1-\beta)S_2) \\
	R_2  \leq   C \left( (1-\beta)S_2 \right)
\end{gathered}
\label{eq:macachiev}
\end{equation}

The otained result \eqref{eq:macachiev} represents the achievability conditions for boundary points of Gaussian MAC. Therefore the proof is given.

\section{Problem 3.20}
\mysection{5.20}{Problem 5.20}
\mysection{6.6}{Problem 6.6}

Let's consider the \textit{Discrete Memoryless Interference Channel} (DM-IC) $p(y_1,y_2|x_1,x_2)$. For a \textit{Successive Cancellation} decoding scheme we need each receiver to decode first the message intended to the other receiver, then decode its message with the additional (hopefully right) knowledge. Let's consider for now fixed probability distributions $p(x,y)=p(x)p(y)$.

Looking at receiver 1, it first needs to decode message $M_2$. To do this, we know from the theory that if $R_2 < I(X_2;Y_1) - \delta(\varepsilon)$, then as $n \rightarrow \infty$ the probability of decoding it incorrectly vanishes. Once it knows the message from transmitter 2, it can decode its message with the additional information. Again, from the theory we know that if $R_1 < I(X_1;Y_1|X_2) - \delta(\varepsilon)$ the message will be asymptotically decoded almost surely. Note that adding the two conditions together we obtain
%
\begin{equation}
R_1+R_2 < I(X_1;Y_1|X_2) + I(X_2;Y_1) = I(X_1,X_2;Y_1)
\end{equation}
%
thus it's useless to add the condition on the sum-rate.

To summarize, the conditions for the correct decoding of receiver 1 are:
%
\begin{equation}
\begin{cases}
	R_2 < I(X_2;Y_1) - \delta(\varepsilon)\\
	R_1 < I(X_1;Y_1|X_2) - \delta(\varepsilon) 
\end{cases}
\end{equation}

Similarly, for receiver 2 we have:
%
\begin{eqnarray}
\begin{cases}
	R_1 < I(X_1;Y_2) - \delta(\varepsilon)\\
	R_2 < I(X_2;Y_2|X_1) - \delta(\varepsilon)
\end{cases}
\end{eqnarray}

Thus, joining everything together and considering different distributions through the time-sharing variable $Q$, we obtain
%
\begin{equation}
\begin{cases}
	R_1 &< \min \qty{ I(X_1;Y_1|X_2,Q),I(X_1;Y_2|Q) }\\
	R_2 &< \min \qty{ I(X_2;Y_1|Q), I(X_2;Y_2|X_1,Q) }\\
	R_1+R_2 &< \min \qty{ I(X_1,X_2;Y_1|Q), I(X_1,X_2;Y_2|Q) }
\end{cases}
\label{eq:succ_canc_bound}
\end{equation}
%
for some distribution $p(q)p(x_1|q)p(x_2|q)$.

Note that the last inequality is redundant, as shown before, but it's now easier for us to compare this with the \textit{Simultaneous Decoding} inner bound of Eq.~\eqref{eq:6.3}.
%
\begin{equation}
\begin{cases}
	R_1 &< \min \qty{ I(X_1;Y_1|X_2,Q),I(X_1;Y_2|X_2,Q) }\\
	R_2 &< \min \qty{ I(X_2;Y_1|X_1,Q), I(X_2;Y_2|X_1,Q) }\\
	R_1+R_2 &< \min \qty{ I(X_1,X_2;Y_1|Q), I(X_1,X_2;Y_2|Q) }
\end{cases}
\tag{6.3}
\label{eq:6.3}
\end{equation}

Since we need to prove that our bound from Eq.~\eqref{eq:succ_canc_bound} is included in the one from Eq.~\eqref{eq:6.3}, it suffices to show that $I(X_1;Y_2|Q) \leq I(X_1;Y_2|X_2,Q)$ and similarly $I(X_2;Y_1|Q) \leq I(X_2;Y_1|X_1,Q)$. This, though, is trivial since we consider independent transmitters (i.e. $p(x_1,x_2)=p(x_1)p(x_2)$) and we know from Chapter~2 of the book that for $X\independent Z$, it holds
%
\begin{equation*}
I(X;Y|Z) \geq I(X;Y)
\end{equation*}
%
which in our case $X=X_1$ and $Z=X_2$.

\end{document}

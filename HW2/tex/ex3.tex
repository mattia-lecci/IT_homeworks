\mysection{4.12}{Problem 4.12}

Since we are interested in creating lists of length $|\LL_2| \leq 2^{nL_2}$, we could think of dividing the original set of messages into equal length lists and sending only the index of the list instead. In other words, we could think of dividing the messages $|\M_2| = 2^{nR_2}$ into $|\K_2| = 2^{nR_2}/2^{nL_2} = 2^{n(R_2-L_2)}$ lists $\qty{ \LL_2^1, \LL_2^2,\, \dots\,,\LL_2^{2^{n(R_2-L_2)}} }$ of $2^{nL_2}$ messages each (assuming $R_2$ and $L_2$ integers just for simplicity). Thus, we can create a function $K_2: \M_2 \rightarrow \K_2$ that returns the index of the list in which a specific message is contained. Just for convenience and without loss of generality, we force $K(1)=1$. Furthermore, we will treat this list number similarly to how we treat normal messages, thus encoding it using an alphabet $\Z_2$ through the function $Z_2^n(k_2) = Z_2^n(K_2(m_2))$. At this point we can easily treat the word $Z_2^n$ like we would treat $X_1^n$ or $X_2^n$.

Like a normal MAC, the transmitters send the couple $(X_1^n(M_1),Z_2^n(K_2))$ (where $K_2 = K_2(M_2)$) and the receiver decodes $(\hat{M}_1,\hat{K}_2)$.

Let's fix $p(x_1,x_2) = p(x_1)p(x_2)$. Note that averaging over all the possible random codebooks and using our notation, we obtain
%
\begin{equation}
\begin{split}
\Pen &= \Pr(M_1 \neq \hat{M}_1 \cup M_2 \notin \LL_2^{\hat{K}_2} | M_1=M_2=1 )\\
&= \Pr(\hat{M}_1\neq 1 \cup \hat{K}_2 \neq 1 | M_1=M_2=1)\\
&\triangleq \Pr(\E | M_1=M_2=1)
\end{split}
\end{equation}

The error event can be decomposed into
%
\begin{align}
\begin{split}
\E ={}& \qty{ \qty(X_1^n(1),Z_2^n(1),Y^n) \notin \typicalitySet }\\
&\cup \{ \exists m_1\neq 1: (X_1^n(m_1),Z_2^n(1),Y^n) \in \typicalitySet \}\\
&\cup \{ \exists m_2\neq 1: (X_1^n(1),Z_2^n(m_2),Y^n) \in \typicalitySet \}\\
&\cup \{ \exists m_1,m_2\neq 1: (X_1^n(m_1),Z_2^n(m_2),Y^n) \in \typicalitySet \}\\
\triangleq{}& \E_1 \cup \E_2 \cup \E_3 \cup \E_4
\end{split}
\end{align}

As usual, we know from the \textit{Law of Large Numbers} that $\E_1 \xrightarrow{n \rightarrow \infty} 0$.

Considering $\E_2$, it's clear that $p(x_1^n,z_2^n,y^n) = p(x_1^n)p(z_2^n)p(y^n|z_2^n)$. Thus, $X_1^n(m1) \independent (Z_2^n(1),Y^n)$, hence
%
\begin{eqnarray}
\Pr(\E_2) \rightarrow 0 & \text{if } R_1 < I(X_1;Z_2,Y) = I(X_1;Z_2) + I(X_1;Y|Z_2) = I(X_1;Y|Z_2)
\end{eqnarray}
%
where the last equality holds from the independence of $X_1^n$ and $Z_2^n$.

Similarly we obtain
%
\begin{align}
\begin{split}
\Pr(\E_3) \rightarrow 0 \quad& \text{if } R_2-L_2 < I(Z_2;X_1,Y) = I(Z_2;Y|X_1)\\
\Pr(\E_4) \rightarrow 0 \quad& \text{if } R_1+R_2-L_2 < I(X_1,Z_2;Y)
\end{split}
\end{align}
%
where the $R_2-L_2$ terms come from the fact that $Z_2^n$ is a function of elements of $\K_2$ which has cardinality equal to $2^{n(R_2-L_2)}$.

Thus, considering all possible probability distributions as well as time sharing, we obtain that $\Pen \rightarrow 0$ if
%
\begin{equation}
(R_1,R_2) \in \conv{\bigcup_{p(x_1),p(z_2)}
\left\{
\begin{array}{rcl}
R_1 &<& I(X_1;Y|Z_2)\\
R_2 &<& I(Z_2;Y|X_1) + L_2\\
R_1+R_2 &<& I(X_1,Z_2;Y) + L_2
\end{array}
\right.
} \label{eq:4.12_achiev_region}
\end{equation}

%%%%%%%%%%%%%%%%%%%%%%%%%%%%%%%%%%%%%%%%%%%%%%%%%%%%%%%%%%%%%%%%%%%
%%%%%%%%%%%%%%%%%%%%%%%%%%%%%%%%%%%%%%%%%%%%%%%%%%%%%%%%%%%%%%%%%%%
%%%%%%%%%%%%%%%%%%%%%%%%%%%%%%%%%%%%%%%%%%%%%%%%%%%%%%%%%%%%%%%%%%%

To prove the converse we need to modify once again Fano's inequality. We define the random variable $E = \indicator[\E]$, where $\indicator$ is the indicator function. Note that $\Pr(E=1) = \Pen$. After some simple manipulation (the same seen in class, not reported here) we obtain that
%
\begin{equation}
H(M_1,M_2|\hat{M}_1,\hat{K}_2) = \ldots \leq 1 + H(M_1,M_2|E,\hat{M}_1,\hat{K}_2)
\end{equation}

Then we can compute
%
\begin{align}
\begin{split}
H(M_1,M_2|E,\hat{M}_1,\hat{K}_2) &= \Pen H(M_1,M_2|E=1,\hat{M}_1,\hat{K}_2)\\
&\hphantom{={}} + (1-\Pen) H(M_1,M_2|E=0,\hat{M}_1,\hat{K}_2)\\
&\leq \Pen \log_2 \qty(2^{nR_1}2^{nR_2}) + (1-\Pen) \log_2 \qty(2^{nL_2})\\
&= \Pen n(R_1+R_2-L_2) + nL_2
\end{split}
\end{align}

Joining everything back in the original equation we obtain
\begin{align}
\begin{split}
H(M_1,M_2|\hat{M}_1,\hat{K}_2) &\leq 1 + H(M_1,M_2|E,\hat{M}_1,\hat{K}_2)\\
&\leq 1 + nL_2 + n\Pen (R_1+R_2-L_2)\\
&= nL_2 + n \qty(\frac{1}{n} + \Pen (R_1+R_2-L_2))\\
&= n (L_2 + \varepsilon_n)
\end{split}
\end{align}
where $\varepsilon_n$ vanishes if $\Pen \rightarrow 0$.

Notice that from the conditional entropy properties and the data processing inequality ($(M_1,M_2)\rightarrow Y^n \rightarrow (\hat{M}_1,\hat{K}_2)$) we have:
%
\begin{equation}
H(M_2|Y^n,M_1) \leq H(M_1,M_2|Y^n) \leq H(M_1,M_2|\hat{M}_1,\hat{K}_2) \leq n(L_2 + \varepsilon_n)
\end{equation}

Thus, following the usual procedure (here not reported)
%
\begin{align}
\begin{split}
nR_2 &\leq H(M_2) = H(M_2|M_1)\\
&= I(M_2;Y^n|M_1) + H(M_2|Y^n,M_1)\\
&\leq I(M_2;Y^n|M_1) + n(L_2 + \varepsilon_n)\\
&\leq \ldots\\
&\leq \sum_{i=1}^n I(X_{2i};Y_i|X_{1i}) + n(L_2 + \varepsilon_n)
\end{split}
\end{align}
\begin{eqnarray}
\begin{split}
\implies R_2 &\leq \frac{1}{n} \sum_{i=1}^n I(X_{2i};Y_i|X_{1i}) + L_2 + \varepsilon_n\\
&\leq I(X_2;Y|X_1,Q) + L_2 + \varepsilon_n
\end{split}
\end{eqnarray}
%
with the classical trick of the time-sharing variable.

In a similar way we can obtain
%
\begin{equation}
R_1 + R_2 \leq I(X_1,X_2;Y|Q) + L_2 + \varepsilon_n
\end{equation}

Lastly, we notice that also $M_1 \rightarrow Y^n \rightarrow \hat{M}_1$, yielding
%
\begin{equation}
H(M_1|Y^n,M_2) \leq H(M_1|Y^n) \leq H(M_1|\hat{M}_1) \leq n\varepsilon_{1,n}
\end{equation}
%
and with similar steps as before we obtain
%
\begin{equation}
R_1 \leq I(X_1;Y|X_2,Q) + \varepsilon_{1,n}
\end{equation}
%
where $\varepsilon_{1,n}$ depends on $\Pr(M_1 \neq \hat{M}_1|M_1=1) \triangleq P_{e1}^{(n)} \leq \Pen$

For all of these we need $\Pen \rightarrow 0$ so to obtain the following region:
%
\begin{equation}
\begin{cases}
R_1 &\leq I(X_1;Y|X_2,Q)\\
R_2 &\leq I(X_2;Y|X_1,Q) + L_2\\
R_1+R_2 &\leq I(X_1,X_2;Y|Q) + L_2
\end{cases} \label{eq:4.12_converse_region}☻
\end{equation}

Similarly to what we've done, we could show that the regions described in Eq.~\ref{eq:4.12_achiev_region} and Eq.~\ref{eq:4.12_converse_region} coincide, thus completed the proof.
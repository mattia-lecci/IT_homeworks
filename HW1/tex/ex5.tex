%\section{Problem 3.20}
\mysection{3.20}{Problem 3.20}

First of all, we are not sure about the power constrain givens by the exercise. We think that either the $1/n$ factor on the LHS term or the $n$ factor on the RHS term should be neglected, giving
%
\begin{equation}\label{eq:myPowerBound}
\frac{1}{n} \sum_{i=1}^n E[x_i^2(U^n)] \leq P
\end{equation}

Thus, for the following results we are going to use Eq.~\eqref{eq:myPowerBound} instead of the one given by the exercise.

\numberwithin{equation}{subsection}
\subsection{Separate source/channel coding}
We are asked to find the minimum distortion achieved by separate source and channel coding.

We know that in order to achieve the lowest possible distortion we need to increase the rate of the source coding. There is a limit to this, though: from Shannon's channel coding theorem we know that exceeding the capacity of the channel will make it impossible to obtain (asymptotically) no errors.

Looking at a single symbol $X_i$ at a time, then, from Eq.~\eqref{eq:myPowerBound} and the channel model $Y=gX+Z$ we obtain $C= \frac{1}{2} \log_2(1+g^2P)$ (well known capacity for Gaussian channel with bound input power).

Since we are looking for the minimum possible distortion we will assume to reach capacity, i.e. $R=C$.

We refer Theorem~3.7 of the book:
%
\begin{theorem}[Source-Channel Separation Theorem]
Given a DMS U and a distortion measure $d(u,\hat{u})$ with rate-distortion function $R(D)$ and a DMC $p(y|x)$ with capacity C, the following statement hold:
\begin{itemize}
	\item If $rR(D)<C$, then $(r,D)$ is achievable
	\item If $(r,D)$ is achievable, then $rR(D) \leq C$
\end{itemize}
Where $r=\frac{\mbox{\#symbols to transmit}}{\mbox{\#transmission to use}}$
\end{theorem}
In our case we use $r=1$, thus we could try to achieve the minimum distortion by setting $R(D)=C$.

From Theorem~3.6 we see that the rate-distortion function for a $WGN(P)$ source with mean squared error distance is
%
\begin{equation}
R(D) = \frac{1}{2} \qty[\log_2 \frac{P}{D}]^+
\end{equation}

Using all of these results we can say
%
\begin{align}\label{eq:3.20a_D}
\begin{split}
&R(D)=C\\
&\frac{1}{2} \qty[\log_2 \frac{P}{D}]^+ = \frac{1}{2} \log_2(1+g^2P) \\
&\frac{P}{D} = 1+g^2P \qquad\qquad \mbox{if } \frac{P}{D} \geq 1\\
&D= \frac{P}{1+g^2P}
\end{split}
\end{align}
%
noticing that $P/D \geq 1 \; \forall g>0$.

\subsection{Linear MMSE}
From MMSE estimator theory we know that
%
\begin{equation}
\hat{X} = E[X|Y] = aY+b
\end{equation}
%
where $a,b$ are real numbers. In fact, for Gaussian estimation, the linear estimator achieves optimal results. It can also easily be written in close form. In the univariate case we have:
%
\begin{equation}
\hat{X} = \frac{\sigma_{XY}}{\sigma_Y} (Y-E[Y]) + E[X] = \frac{\sigma_{XY}}{\sigma_Y^2} Y
\end{equation}
%
since $X=U$ comes from the $WGN(P)$ source (by definition it has zero mean), and $E[Y]=E[gX+Z] = gE[X]+E[Z] = 0$. It is also simple to compute $\sigma_Y^2 = E[(Y-E[Y])^2] = E[Y^2] = g^2P+1$ since $X\independent Y$. Similarly, $\sigma_{XY} = E[(X-E[X])(Y-E[Y])] = E[XY] = E[gX^2+XZ] = gP$.\\
Hence, we can write $\hat{X} = \frac{gP}{1+g^2P}Y$.

Finally, we can write the distortion as
%
\begin{align}
\begin{split}
D =& E[d(X,\hat{X})] = E[(X-\hat{X})^2] = E \qty[\qty(X-\frac{gP}{1+g^2P}Y)^2]\\
=& E \qty[\qty( X \qty(1-\frac{g^2P}{1+g^2P}) - \frac{gP}{1+g^2P}Z)^2]\\
=& E \qty[\qty( \frac{1}{1+g^2P}X)^2] + E\qty[\qty( \frac{gP}{1+g^2P}Z)^2] - 2E\qty[ \frac{1}{1+g^2P}X \frac{gP}{1+g^2P}Z ]\\
=& \frac{P}{(1+g^2P)^2} + \frac{g^2P^2}{(1+g^2P)^2} = \frac{P(1+g^2P)}{(1+g^2P)^2} = \frac{P}{1+g^2P}
\end{split}
\end{align}

Note that it's the same as the lower bound on the distortion from Eq.~\eqref{eq:3.20a_D}. Since doing source coding on a Gaussian (thus continuous) source and then perform channel coding requires a lot of effort, sending the uncoded signal with MMSE receiver is the optimal choice both in term of complexity and distortion.

\numberwithin{equation}{section}

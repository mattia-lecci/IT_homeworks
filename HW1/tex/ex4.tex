%\section*{Problem 3.14}
\mysection{3.14}{Problem 3.14}

\newcommand{\typical}{\ensuremath{\mathcal{T}_\varepsilon^{(n)}}}
\newcommand{\typicalLong}{\ensuremath{\mathcal{T}_\varepsilon^{(n)}(X^n,Y^n)}}

We decided to use the following coding/decoding scheme: the coder simply encodes the message $M$ into the codeword $X^n(M)$ like in the normal case seen in class. The decoder, instead, looks at all possible codewords $X^n(m) \; \forall m\in \mathcal{M}$ and looks whether or not they are typical and creates $\mathcal{L}(Y^n) = \typicalLong$ if $|\typical| \leq 2^{nL}$ or $\mathcal{L}(Y^n)=\emptyset$ if $|\typical| > 2^{nL}$.

\numberwithin{equation}{subsection}
\subsection{Achievability}
Using the coder/decoder description given at the beginning of the section, the error probability can be written as
%
\begin{align}
\begin{split}
P_e^{(n)}=&E_\mathcal{C}[P_{e,\mathcal{C}}^n] = \Pr( M\notin \mathcal{L}(Y^n) |M=1 )\\
=& \Pr( \{(X^n(1),Y^n)\notin\typical| M=1\} \cup \{ |\typical|>2^{nL} \}|M=1)\\
\leq& \Pr((X^n(1),Y^n)\notin\typical|M=1) + \Pr(|\typical|>2^{nL}|M=1)\\
\triangleq& \Pr(\mathcal{E}_1) + \Pr(\mathcal{E}_2)
\end{split}
\end{align}

As seen in class, by the conditional typicality lemma $P(\mathcal{E}_1) \xrightarrow{n\rightarrow\infty}0$. All the effort, once again, will be focused on creating appropriate bounds for $\Pr(\mathcal{E}_2)$ as follows
%
\begin{align} \label{eq:3.14_E2}
\begin{split}
\Pr(\mathcal{E}_2) =& \Pr(|\typicalLong|>2^{nL}|M=1)\\
 \leq& \Pr(|\typicalLong| \geq 2^{nL}|M=1)\\
\leq& \frac{E \qty[\qty|\typicalLong| \,|M=1]}{2^{nL}}
\end{split}
\end{align}
%
where the last inequality follows from Markov's Inequality.

Now, we can bound the mean cardinality of the typicality set as follows:
%
\begin{align} \label{eq:3.14_expectation_bound}
\begin{split}
E \qty[\qty|\typicalLong| \,|M=1] =& E\qty[ \sum_{m=1}^{2^{nR}} \indicator((X^n(m),Y^n) \in \typical) |M=1 ]\\
=& \sum_{m=1}^{2^{nR}} E[\indicator((X^n(m),Y^n)\in \typical |M=1)]\\
=& \sum_{m=1}^{2^{nR}} \Pr((X^n(m),Y^n)\in \typical |M=1)\\
\leq& \Pr((X^n(1),Y^n)\in \typical |M=1) +\\
&\sum_{m=2}^{2^{nR}} \Pr((X^n(m),Y^n)\in \typical |M=1)\\
\leq& 1 + (2^{nR}-1)2^{-n(I(X;Y)-\delta(\varepsilon))}\\
\leq& 1 + 2^{-n(C-R - \delta(\varepsilon))}
\end{split}
\end{align}

Now, joining Eqs.~\eqref{eq:3.14_expectation_bound},~\eqref{eq:3.14_E2} we obtain
%
\begin{equation}
\Pr(\mathcal{E}_2) \leq 2^{-nL} + 2^{-n(C+L-R - \delta(\varepsilon))}
\end{equation}
%
which tends to zero if $R<C+L$.

\subsection{Converse}
We will start showing the modified Fano's Inequality.

We are interested in the error probability defined as $P_e^{(n)} = \Pr(M \notin \mathcal{L}(Y^n))$. For this, similarly to the proof of Fano's Inequality seen in class, we define the random variable $\mathcal{E}$ such that $\mathcal{E}=0$ if $M \in \mathcal{L}$, $\mathcal{E}=1$ if $m \notin \mathcal{L}$. This implies that $E[\mathcal{E}] = P_e^{(n)}$ and $H(\mathcal{E}) = H(P_e^{(n)})$. Again, following the same procedure we obtain
%
\begin{equation}
H(M|\mathcal{L}) = H(\mathcal{E}|\mathcal{L})
+H(M|\mathcal{L},\mathcal{E})
-H(\mathcal{E}|M,\mathcal{L})
\end{equation}

Now, clearly $H(E|M,\mathcal{L})=0$ since once we know the message $M$ and the list $\mathcal{L}$, we deterministically know the value of $\mathcal{E}$. Furthermore, $H(\mathcal{E}|\mathcal{L}) \leq H(\mathcal{E}) = H(P_e^{(n)})$ due to the conditional entropy property. Finally, we can write $H(M|\mathcal{E},\mathcal{L}) = P_e^{(n)} H(M|\mathcal{E}=1,\mathcal{L}) + (1-P_e^{(n)})H(M|\mathcal{E}=0,\mathcal{L})$. Now, if $\mathcal{E}=0$ it means that $M \in \mathcal{L}$, thus knowing both this and $\mathcal{L}$ means that $M$ could be any of the messages in the list, i.e. $H(M|\mathcal{L},\mathcal{E}=0) = \log_2|\mathcal{L}|$. Similarly, if $M \notin \mathcal{L}$ it must be one of the other $|\mathcal{M}|-|\mathcal{L}|$ messages, i.e. $H(M|\mathcal{L},M\notin \mathcal{L})=\log_2(|\mathcal{M}|-|\mathcal{L}|)$.

Bringing all together we get
%
\begin{align}
\begin{split}
H(M|\mathcal{L}) \leq & H(Pe) + P_e^{(n)}\log_2(|\mathcal{M}|-|\mathcal{L}|) + (1-P_e^{(n)})\log_2|\mathcal{L}|\\
=& H(P_e^{(n)}) + \log_2|\mathcal{L}| + P_e^{(n)}\log_2\qty(\frac{|\mathcal{M}|-|\mathcal{L}|}{|\mathcal{L}|})\\
=& H(P_e^{(n)}) + \log_2|\mathcal{L}| + P_e^{(n)}\log_2\qty(\frac{|\mathcal{M}|}{|\mathcal{L}|} - 1)\\
\leq& 1 + \log_2|\mathcal{L}| + P_e^{(n)}\log_2\frac{|\mathcal{M}|}{|\mathcal{L}|}
\end{split}
\end{align}

In our case we have $\mathcal{M}=2^{nR}$ and $1\leq |\mathcal{L}| \leq 2^{nL}$, thus we can upper bound $\log_2|\mathcal{L}| \leq nL$ and $\log_2(|\mathcal{M}|/|\mathcal{L}|) \leq nR$, thus yielding
%
\begin{equation}
H(M|\mathcal{L}) \leq 1+nL+P_e^{(n)} \, nR = nL + n\qty(\frac{1}{n} + P_e^{(n)} R) = nL + n\varepsilon_n
\end{equation}

Now the proof of the converse is exactly the same as the one seen in class, using this modified version of Fano's Inequality:
%
\begin{align}
\begin{split}
nR =& H(M)\\
=&  I(M;Y^n) + H(M|Y^n)\\
\stackrel{(a)}{\leq}& I(M;Y^n) + nL + n\varepsilon_n\\
=& \sum_{i=1}^n I(M;Y_i|Y^{i-1}) + nL+n\varepsilon_n\\
\stackrel{(b)}{=}& \sum_{i=1}^n I(M,Y^{i-1};Y_i) + nL+n\varepsilon_n\\
\stackrel{(c)}{=}& \sum_{i=1}^n I(M,X_i,Y^{i-1};Y_i) + nL+n\varepsilon_n\\
\stackrel{(d)}{=}& \sum_{i=1}^n I(X_i;Y_i) + nL+n\varepsilon_n\\
\leq& nC + nL+n\varepsilon_n
\end{split}
\end{align}
%
thus $R\leq C+L$ since $\varepsilon_n \xrightarrow{n\rightarrow\infty} 0$. In $(a)$ we used the fact that $\mathcal{L} \independent M |Y^n$ (or in Markov chain notation $M\rightarrow Y^n\rightarrow \mathcal{L}$), thus $H(M|Y^n)\leq H(M|\mathcal{L}) \leq nL + n\varepsilon_n$. In $(b)$ we used the conditioning rule $I(M,Y^{i-1};Y_i) = I(Y^{i-1};Y_i) + I(M;Y_i|Y^{i-1}) \geq I(M;Y_i|Y^{i-1})$. In $(c)$ we know that $X_i$ is a deterministic function of $M$ so it doesn't add or subtract any information. In $(d)$ we use exploit the memoryless property of the channel and the conditional independence $Y_i \independent (M,Y^{i-1})|X_i$.

\numberwithin{equation}{section}
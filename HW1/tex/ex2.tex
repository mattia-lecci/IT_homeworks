\section{Problem 3.13}

Is given a Gaussian product channel $Y_j = g_j \cdot X_j + Z_j$ $j\in \{1,2\}$ with $g_1 < g_2$ and average power constraint $P$. We want to know above what power $P^*$ we start to use both the channels and what are the features of the \textit{energy-per-bit-rate function} $E_b(R)$.

\subsection{Optimal power allocation}


We notiche that to achieve the maximum capacity from the channel we have to solve the following optimization problem.

\begin{equation}
\begin{gathered}
	\max_{P_j} \sum_{j=1}^2 \frac{1}{2} \cdot \log(1+g_j^2 \cdot P_j) \quad subject \quad to\\
	-P_j \neq 0 \quad j\in \{1,2\} \\
	\sum_{j=1}^2 P_j - P = 0
\end{gathered}
\label{optproblem}
\end{equation}

In order to solve \eqref{optproblem} we build the following equation.

\begin{equation}
	\nabla_{P_j}\Big\{-\sum_{j=1}^2 \frac{1}{2} \cdot \log(1+g_j^2 \cdot P_j) + \sum_{j=1}^2 \lambda_j \cdot (-P_j) + \nu \cdot \Big(\sum_{j=1}^2 P_j - P\Big)  \Big\} = 0
	\label{f0}
\end{equation}

Solving \eqref{f0} we obtain what follows.

\begin{equation}
	\begin{gathered}
		\frac{1}{2} \cdot \frac{1}{P_j+\frac{1}{g_j^2}}+\lambda_j-\nu=0 \\
		\Rightarrow \frac{1}{2} \cdot \frac{1}{P_j+\frac{1}{g_j^2}} \leq \nu = \frac{1}{2\mu}
	\end{gathered}
\end{equation}

\begin{equation}
	\begin{gathered}
		\sum_{j=1}^2 \max\Big\{\mu - \frac{1}{g_j^2},0\Big\} = P \\
		\Rightarrow \mu = \frac{P}{2}+\frac{1}{2g_1^2} + \frac{1}{2g_2^2}
	\end{gathered}
\end{equation}

\begin{equation}
	\begin{gathered}
	 P_j = \max\Big\{\mu - \frac{1}{g_j^2},0\Big\} \\
	 \Rightarrow P_1= \max\Big\{\frac{P}{2} + \frac{1}{2g_2^2} - \frac{1}{2g_1^2},0\Big\} = \begin{cases}
	  \frac{P}{2} + \frac{1}{2g_2^2} - \frac{1}{2g_1^2} \quad P \geq \frac{1}{g_1^2} - \frac{1}{g_2^2} \\
		0 \quad P < \frac{1}{g_1^2} - \frac{1}{g_2^2}
	 \end{cases} \\
	 \Rightarrow P_2= \max\Big\{\frac{P}{2} + \frac{1}{2g_1^2} - \frac{1}{2g_2^2},0\Big\} = \frac{P}{2} + \frac{1}{2g_1^2} - \frac{1}{2g_2^2} \quad \forall \quad P \geq 0
	\end{gathered}
\end{equation}

We conclude that the second channel is opened for every amount of allocated power $P$ while the first channel is opened only if and only if \eqref{Pcondition} is verified.

\begin{equation}
	P \geq \frac{1}{g_1^2} - \frac{1}{g_2^2}
	\label{Pcondition}
\end{equation}

\subsection{Energy-per-bit-rate function computation}

Now we want to compute the \textit{energy-per-bit-rate function} $E_b(R)$ in the scenario previously described. We know that $P = R \cdot E$ where $R$ is the \textit{bit-rate} of the channel and $E$ is the used energy for every bit. In addiction we know that depending of what amount of power is avaliable, the \textit{bit-rate} is restricted as shown in \eqref{1case} and \eqref{2case}.

\begin{equation}
		R \leq \frac{1}{2} \cdot \log\big( 1 + g_2^2P \big) \Leftrightarrow P < \frac{1}{g_1^2} - \frac{1}{g_2^2}
		\label{1case}
\end{equation}


\begin{equation}
		R \leq \frac{1}{2} \cdot \sum_{j=1}^2 \log\Big(\frac{g_j^2}{2} \cdot \Big(P+\frac{1}{g_1^2} + \frac{1}{g_2^2}\Big)\Big) \Leftrightarrow P \geq \frac{1}{g_1^2} - \frac{1}{g_2^2}
		\label{2case}
\end{equation}

Supponing to achieve the maximum rate in both \eqref{1case} and \eqref{2case} we can compute R in function of P and then write again the boundaries condition as follows.

\begin{equation}
	\begin{gathered}
		R \leq \frac{1}{2} \cdot \log\big( 1 + g_2^2P \big) \\
		\Rightarrow \frac{2^{2R}-1}{g_2^2} \leq P < \frac{1}{g_1^2} - \frac{1}{g_2^2} \\
		\Rightarrow R < \frac{1}{2} \log\Big(\frac{g_2^2}{g_1^2}\Big)
	\end{gathered}
	\label{bound1}
\end{equation}

\begin{equation}
	\begin{gathered}
		R \leq \frac{1}{2} \cdot \sum_{j=1}^2 \log\Big(\frac{g_j^2}{2} \cdot \Big(P+\frac{1}{g_1^2} + \frac{1}{g_2^2}\Big)\Big) \\
		\Rightarrow P \geq \frac{2^{R+1}}{g_1g_2} -\frac{1}{g_1^2} - \frac{1}{g_2^2}
	\end{gathered}
	\label{bound2}
\end{equation}


Let's first consider the case where only one channel is opened. Sobstituting $P=E \cdot R$ in \eqref{1case} we get what follows.

\begin{equation}
	\begin{gathered}
		R \leq \frac{1}{2} \cdot \log\big( 1 + g_2^2 E R \big) \\
		\Rightarrow E \geq \frac{1}{R}\Big(\frac{2^{2R}-1}{g_2^2}\Big)
	\end{gathered}
\end{equation}

Let's now consider the case where both the channels are opened. Sobstituiting $P=E \cdot R$ in \eqref{2case} we get what follows.

\begin{equation}
	\begin{gathered}
		R \leq \frac{1}{2} \cdot \sum_{j=1}^2 \log\Big(\frac{g_j^2}{2} \cdot \Big(ER+\frac{1}{g_1^2} + \frac{1}{g_2^2}\Big)\Big) \\
		\Rightarrow R \leq \frac{1}{2} \cdot \log\Big(\frac{g_1^2 g_2^2}{4} \cdot \Big(ER+\frac{1}{g_1^2} + \frac{1}{g_2^2}\Big)^2\Big) \\
		\Rightarrow E \geq \frac{1}{R} \Big(\frac{2^{R+1}}{g_1g_2} -\frac{1}{g_1^2} - \frac{1}{g_2^2} \Big)
	\end{gathered}
\end{equation}

We conclude that \textit{energy-per-bit rate function} is equal to the minimum energy-per-bit required for every value of R as we can see from \eqref{epbfunction}.

\begin{equation} E_b(R)=
	\begin{cases}
		\frac{1}{R}\Big(\frac{2^{2R}-1}{g_2^2}\Big) \quad P < \frac{1}{g_1^2} - \frac{1}{g_2^2}  \\
		\frac{1}{R} \Big(\frac{2^{R+1}}{g_1g_2} -\frac{1}{g_1^2} - \frac{1}{g_2^2} \Big)  \quad P \geq \frac{1}{g_1^2} - \frac{1}{g_2^2}
	\end{cases}
	\label{epbfunction}
\end{equation}

Now we want to compute the minimum energy-per-bit as $R \rightarrow 0$.


\begin{equation}
	\lim_{R \rightarrow 0} E_b(R) =
		\lim_{R \rightarrow 0} \Big(\frac{2^{2R}-1}{g_2^2 R}\Big) = \lim_{R \rightarrow 0} \frac{ \frac{d(2^{2R}-1)}{dR}} {\frac{dg_2^2 R}{dR}}=\frac{2\ln{2}}{g_2^2} \quad P < \frac{1}{g_1^2} - \frac{1}{g_2^2}
\end{equation}

\begin{equation}
		\lim_{R \rightarrow 0} E_b(R) =
		\lim_{R \rightarrow 0} \frac{1}{R} \Big(\frac{2^{R+1}}{g_1g_2} -\frac{1}{g_1^2} - \frac{1}{g_2^2} \Big) = \lim_{R \rightarrow 0} \frac{ \frac{d(2^{R+1})}{dR}} {\frac{dg_1g_2 R}{dR}}=\frac{2\ln{2}}{g_1g_2}  \quad P \geq \frac{1}{g_1^2} - \frac{1}{g_2^2}
\end{equation}

We conclude that the minimum energy-per-bit as $R \rightarrow 0$ is equal to \eqref{emin}.

\begin{equation}
	\lim_{R \rightarrow 0} E_b(R) = \begin{cases}
		\frac{2\ln{2}}{g_2^2} \quad P < \frac{1}{g_1^2} - \frac{1}{g_2^2} \\
		\frac{2\ln{2}}{g_1g_2} \quad P \geq \frac{1}{g_1^2} - \frac{1}{g_2^2}
\end{cases}
\label{emin}
\end{equation}
